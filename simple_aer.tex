\documentclass{article}
\usepackage{mystyle}

\begin{document}
\title{Simple AER}
\author{Sam Fok}
\maketitle

In this document, we develop a simple address-event representation (AER) system for an array of $n$ neurons.

\begin{csp}
NT(n)\equiv\langle\pll\!i:0..n\-1:*[[P;P]]\rangle
NR(n)\equiv\langle\pll\!i:0..n\-1:*[[P;P]]\rangle
\end{csp}
%%%%%%%%%%%%%%%%%%%%%%%%%%%%%%%%%%%%%%%%%%%%%%%%%%%%%%%%%%%%%%%%%%%%%%%%%%%%%%%
\part{Transmitter ($AEXT$)}

When a neuron spikes, the transmitter sends the neuron's addresss.

%%%%%%%%%%%%%%%%%%%%%%%%%%%%%%%%%%%%%%%%%%%%%%%%%%%%%%%%%%%%%%%%%%%%%%%%%%%%%%%
\section{Transmitter Decomposition}

The highest level specification of the transmitter is given by

\begin{csp}
AEXT(n)\equiv
  *[[\langle\|\!i:0..n\-1:#{C`i}->T!i;C`i;T!i;C`i]]
\end{csp}

\noindent We decompose $AEXT$ into control $CTRL$ and data $DATA$ processes.

\begin{csp}
AEXT(n)\equiv*[CTRL(n)\pll\!DATA(n)]
\end{csp}

\begin{csp}
CTRL(n)\equiv
  *[[\langle\|i:0..n\-1:#{C`i}->C`i;C`i]]
\end{csp}

\begin{csp}
DATA(n)\equiv
  \langle\pll\!i:0..n\-1:*[[C`i;A`i;T!i;T!i;A`i;C`i]]\rangle
\end{csp}

$CTRL$ is just an $n$-way arbiter, which we develop elsewhere.

\begin{csp}
TX(n)\equiv
  *[[\langle[]i:0..n\-1:#{S`i}->T!i;S`i;T!i;S`i\rangle]]
\end{csp}

%%%%%%%%%%%%%%%%%%%%%%%%%%%%%%%%%%%%%%%%%%%%%%%%%%%%%%%%%%%%%%%%%%%%%%%%%%%%%%%
\part{Receiver ($AERV$)}

%%%%%%%%%%%%%%%%%%%%%%%%%%%%%%%%%%%%%%%%%%%%%%%%%%%%%%%%%%%%%%%%%%%%%%%%%%%%%%%
\section{Receiver Decomposition}

The highest level of the receiver is given by

\begin{csp}
AERV(n)\equiv*[[\langle[]i:0..n\-1:C`i;C`i]]
\end{csp}
\end{document}
