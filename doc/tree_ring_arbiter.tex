\documentclass{article}
\usepackage{mystyle}

\begin{document}

\title{Arbitration}
\maketitle

%%%%%%%%%%%%%%%%%%%%%%%%%%%%%%%%%%%%%%%%%%%%%%%%%%%%%%%%%%%%%%%%%%%%%%%%%%%%%%%
\section{Greedy but Fair N-way Arbiter using Tree/Ring Sequencing}

The tree is built from NODEs. 
The leaves of the tree service requests from $N$ clients.

\begin{figure}
  \centering
  \includegraphics[width=.5\textwidth]{img/tree_ring_arbiter.pdf}
  \caption{Greedy, fair, tree/ring, N-way arbiter NODE}
  \label{fig:n_arb}
\end{figure}

\subsubsection*{Accounting}

NODES cost 30$K$+18 transistors, where $K$ is the radix of the tree.

\begin{center}
    \begin{tabular}{|r|l|}
    \hline
    Configuration & transistors \\ \hline
    $K$-ary tree & $(30K+18)\frac{N-1}{K-1}$ \\ \hline
    Flat ring & $30N+18$ \\ \hline
    Binary tree & $78(N-1)$ \\ \hline
    4-ary tree & $46(N-1)$ \\ \hline
    \end{tabular}
\end{center}

%%%%%%%%%%%%%%%%%%%%%%%%%%%%%%%%%%%%%%%%%%%%%%%%%%%%%%%%%%%%
\subsection{NODE}

For a binary tree,

\begin{csp}
*[[#{C`0}|#{C`1}];P\*(
  [#{C`0}->C`0;C`0\|~#{C`0}->skip];
  [#{C`1}->C`1;C`1\|~#{C`1}->skip])
 ]
\end{csp}

\noindent
For a $K$-ary tree,

\begin{csp}
*[[\langle|k:K:#{C`k}\rangle];
  P\*\langle;k:K:[#{C`0}->C`0;C`0\|~#{C`0}->skip]\rangle
 ]]
\end{csp}

\noindent
We decompose NODE into PRODUCER and CONSUMER processes.

\subsubsection*{Accounting}

\begin{center}
    \begin{tabular}{|r|l|l|l|}
    \hline
    component & transistors/component & components/NODE & transistors/NODE \\ \hline
    PRODUCER & 18+$K$ & 1 & 18+$K$ \\ \hline
    CONSUMER & 29 & $K$ & 29$K$ \\ \hline
    \hline \multicolumn{3}{|r|}{total transistors/NODE} & 30$K$+18 \\ \hline
    \end{tabular}
\end{center}

%%%%%%%%%%%%%%%%%%%%%%%%%%%%%%%%%%%%%%%%%%%%%%%%%%%%%%%%%%%%
\subsection{PRODUCER}

PRODUCER produces parent requests upon sensing child requests
and sequences between CONSUMERs.
For a binary tree,

\begin{csp}
*[[#{C`0}|#{C`1}];P\*\langle\*K:k:S`k\rangle]]
\end{csp}

\noindent
For a $K$-ary tree,

\begin{csp}
*[[\langle|K:k:#{C`k}\rangle];P\*\langle\*K:k:S`k\rangle]]
\end{csp}

\noindent 
We decompose PRODUCER into CTRL and SEQ.

\subsubsection*{Accounting}

\begin{center}
    \begin{tabular}{|r|l|l|l|}
    \hline
    component & transistors/component & components/PRODUCER & transistors/PRODUCER \\ \hline
    CTRL & 18+$K$ & 1 & 18+$K$ \\ \hline
    SEQ & 0 & 1 & 0 \\ \hline
    \hline \multicolumn{3}{|r|}{total transistors/PRODUCER} & 18+$K$ \\ \hline
    \end{tabular}
\end{center}

%%%%%%%%%%%%%%%%%%%%%%%%%%%%%%%%%%%%%%%%%%%%%%%%%%%%%%%%%%%%
\subsection{CTRL}

CTRL relays the child requests to the parent 
and initiates the sequencing between CONSUMERS.

\subsubsection*{CHP}

For a binary tree,

\begin{csp}
*[[#{C`0}|#{C`1}];P\*S]
\end{csp}

\noindent For a $K$-ary tree,

\begin{csp}
*[[\langle|K:k:#{C`k}\rangle];P\*S]
\end{csp}

\subsubsection*{HSE}

For a binary tree,

\begin{hse}
*[[pe];[c0r|c1r];x-;
  pr+;[~pe];
  sr+;[sa];x+;sr-;[~sa];
  pr-]
\end{hse}

\subsubsection*{PRS}

\begin{prs2}
pe & (c0r | c1r) -> x-
sa -> x+
\end{prs2}

\begin{prs2}
~x | sa -> pr+
x & ~sa -> pr-
\end{prs2}

\begin{prs2}
~x & ~pe -> sr+
x | pe -> sr-
\end{prs2}

\subsubsection*{CMOS-implementable PRS}

\begin{prs2}
_pe & (c0r | c1r) -> x-
~_sa -> x+
\end{prs2}

\begin{prs2}
~x | ~_sa -> pr+
x & _sa -> pr-
\end{prs2}

\begin{prs2}
~x & ~_pe -> sr+
x | _pe -> sr-
\end{prs2}

\begin{prs2}
pe -> _pe-
~pe -> _pe+

sa -> _sa-
~sa -> _sa+
\end{prs2}

\subsubsection*{Accounting}

\begin{center}
    \begin{tabular}{|r|l|l|}
    \hline
    rule & transistor count & comments \\ \hline
    $x$ & 2+$K$+4 & \\ \hline
    $pr$ & 4 & \\ \hline
    $sr$ & 4 & \\ \hline
    $pe$ & 2 & \\ \hline
    $_sa$ & 2 & \\ \hline
    \hline total & 18+$K$ & \\ \hline
    \end{tabular}
\end{center}

%%%%%%%%%%%%%%%%%%%%%%%%%%%%%%%%%%%%%%%%%%%%%%%%%%%%%%%%%%%%
\subsection{SEQ}

SEQ sequences through the consumers.

\subsubsection*{CHP}

In general,

\begin{csp}
SEQ(K)\equiv
*[SP\*\langle\*k:K:SC`k\rangle]
\end{csp}

\noindent
For a binary tree,

\begin{csp}
*[SP\*SC`0\*SC`1]
\end{csp}

\noindent
We build a $K$-way sequencer as $K'$-way sequencer tree defined recursively as

\begin{csp}
SEQ(K)\equiv\!SEQ(K')\pll\langle\pll\!k':K'\-1:SEQ(K/K')\rangle\pll\!SEQ(K\-(K'\-1)K/K')
\end{csp}

\noindent where the recursion ends when $K\le K'$. For example, a binary tree
would be defined recursively as

\begin{csp}
SEQ(K) = SEQ(2)\pll\!SEQ(K/2)\pll\!SEQ(K\-K/2)
\end{csp}

\noindent
The $SC$ channels of the parent sequencer are the $SP$ channels of the the child 
sequencers. The $SC$ channels of the leaf sequencers connect to the consumers.

\subsubsection*{HSE}

For a binary tree,

\begin{hse}
*[[spr];
  sc`0r+;[sc`0a];sc`1r+;[sc`1a];
  spa+;[~spr];
  sc`0r-;[~sc`0a];sc`1r-;[~sc`1a];
  spa-
 ]
\end{hse}

\noindent
In general,

\begin{hse}
*[[spr];
  \langle;k':K':sc`k'r+;[sc`k'a]\rangle
  spa+;[~spr];
  \langle;k':K':sc`k'r-;[~sc`k'a]\rangle
  spa-
 ]
\end{hse}

\subsubsection*{PRS}

\begin{prs2}
spr -> sc0r+
~spr -> sc0r-

sc(k'\-1)r -> spa+
~sc(k'\-1)r -> spa-
\end{prs2}

\begin{prs2}
sck'a -> sc(k'\+1)r+
~sck'a -> sc(k'\+1)r-
\end{prs2}

\noindent
These are just wires, so all $K'$-way tree sequencers that implement a 
$K$-way sequencer are equivalent.

%%%%%%%%%%%%%%%%%%%%%%%%%%%%%%%%%%%%%%%%%%%%%%%%%%%%%%%%%%%%
\subsection{CONSUMER}

CONSUMER checks and services requests from children. 

\subsubsection*{CHP}

\begin{csp}
*[[#C->S;C
  \|#S->S]]
\end{csp}

\noindent
This is just a variation on the precise exceptions circuit.

\subsubsection*{HSE}

\begin{hse}
*[[cr->[sr];sa+;[~sr];ce-;[~cr];sa-;ce+
  \|sr->sa+;[~sr];sa-
 ]]
\end{hse}

\noindent
We break out nondeterministic selection.

\begin{hse}
*[[cr->c+;[~cr];c-
  \|sa->s+;[~sa];s-
 ]]
*[[c->[sr];sta+;[~sr];ce-;[~c];sta-;ce+
  []s->sfa+;[~s];sfa-
 ]]
*[[sta|sfa];sa+;[~sta&~sfa];sa-]
\end{hse}

\subsubsection*{PRS}

\begin{prs2}
c & sr -> sta+
~c -> sta-

s -> sfa+
~s -> sfa-
\end{prs2}

\begin{prs2}
~sta | sr -> ce+
sta & ~sr -> ce-
\end{prs2}

\begin{prs2}
sta | sfa -> sa+
~sta & ~sfa -> sa-
\end{prs2}

\subsubsection*{CMOS-implementable PRS}

\begin{prs2}
c & sr -> _sta-
~c -> _sta+

s -> _sfa-
~s -> _sfa+
\end{prs2}

\begin{prs2}
_sta | sr -> _ce-
~_sta & ~sr -> _ce+
\end{prs2}

\begin{prs2}
~_sta | ~_sfa -> sa+
_sta & _sfa -> sa-
\end{prs2}

\subsubsection*{Accounting}

\begin{center}
    \begin{tabular}{|r|l|l|}
    \hline
    rule & transistor count & comments \\ \hline
    $[c,s]$ & 12 & 2-way active-low arbiter \\ \hline
    $\_sta$ & 7 & \\ \hline
    $\_sfa$ & 2 & \\ \hline
    $\_ce$ & 4 & \\ \hline
    $sa$ & 4 & \\ \hline
    \hline total & 29 & \\ \hline
    \end{tabular}
\end{center}

%%%%%%%%%%%%%%%%%%%%%%%%%%%%%%%%%%%%%%%%%%%%%%%%%%%%%%%%%%%%%%%%%%%%%%%%%%%%%%%
\end{document}
