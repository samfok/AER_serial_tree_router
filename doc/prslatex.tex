% prstex.tex:
%	This file is loaded by prs.tex/prs.sty for use with
%	latex, latex2e, latex-nfss (=nfss1).
%	See prs.doc for documentation of the prs maros.
%
% copyright (c) 1995 California Institute of Technology
% Author: Marcel R. van der Goot

% Update history:
%
% version 2.0: For use with prs.tex version 2.0
%	release 26 July 1995: First version of this file (before, it was part
%		of prs.tex).
%
%%%%%%%%%%%%%%%%%%%%%%%%%%%%%%%%%%%%%%%%%%%%%%%%%%%%%%%%%%%%%%%%%%%%%%%%%%%%%%%

%%%%%%%%%% prs environment %%%%%%%%%%%%%%%%%%%%%%%%%%%%%%%%%%%%%%%%%%%%%%%%%%%%

\let\tmp=\theprs
\newcounter{prs}
\def\incrprsnr{\refstepcounter{prs}}
\let\prsnr=\c@prs
\let\theprs=\tmp

{\catcode`\*=12 % because the * must be read in at-mode

\gdef\def_end_prs#1#2%
   {\ifnum#1>1
	 \ifnum#2>0
	      \def\enddolines{\end{prs2*}}
	 \else\def\enddolines{\end{prs2}}
	 \fi
    \else\ifnum#2>0
	      \def\enddolines{\end{prs*}}
	 \else\def\enddolines{\end{prs}}
	 \fi
    \fi
    \let\finishdolines=\enddolines
   }
}

\newenvironment{prs}{\begin_prs10}{\end_prs}
\newenvironment{prs*}{\begin_prs11}{\end_prs}
\newenvironment{prs2}{\begin_prs20}{\end_prs}
\newenvironment{prs2*}{\begin_prs21}{\end_prs}

% The spacing for environments in LaTeX is normally obtained with
% \trivlist\item[]\leavevmode. We need to pack the result in an extra vbox
% because in \endprs_onecol there's an \unvbox, which would cause a switch
% from horizontal to vertical mode, inserting a \par and extra space. You
% cannot omit the \leavevmode, because LaTeX will complain about a missing
% item. Note that the extra vbox prevents page breaks.

\def\beforeprs{\trivlist\item[]\leavevmode\vbox\bgroup}
\def\afterprs{\egroup\endtrivlist}

%%%%%%%%%% hse and csp environments %%%%%%%%%%%%%%%%%%%%%%%%%%%%%%%%%%%%%%%%%%%

\let\tmp=\thehse
\newcounter{hse}
\def\incrhsenr{\refstepcounter{hse}}
\let\hsenr=\c@hse
\let\thehse=\tmp

{\catcode`\*=12 % because the * must be read in at-mode

\gdef\def_end_hse#1#2%
   {\ifnum#1>0
	 \ifnum#2>0
	      \def\enddolines{\end{csp*}}
	 \else\def\enddolines{\end{csp}}
	 \fi
    \else\ifnum#2>0
	      \def\enddolines{\end{hse*}}
	 \else\def\enddolines{\end{hse}}
	      \he_compat % he environment, for compatibility.
	 \fi
    \fi
    \let\finishdolines=\enddolines
   }
}

\let\he_compat=\relax

\newenvironment{hse}{\begin_hse00}{\end_hse}
\newenvironment{hse*}{\begin_hse01}{\end_hse}

% for compatibility with the old version:
\newenvironment{he}{\typeout{WARNING: the he environment is now called hse}%
		    \def\he_compat{\def\enddolines{\end{he}}}%
		    \begin_hse00}%
		   {\end_hse}

\let\beforehse=\beforeprs
\let\afterhse=\afterprs

\newenvironment{csp}{\begin_hse10}{\end_hse}
\newenvironment{csp*}{\begin_hse11}{\end_hse}

%%%%%%%%%% other %%%%%%%%%%%%%%%%%%%%%%%%%%%%%%%%%%%%%%%%%%%%%%%%%%%%%%%%%%%%%%

\def\makeatother{\useatmode}


% In LaTeX2e, \useatmode and \prsdeftt must be done after \begin{document}.
% Even if they have been done earlier, apparently \begin{document} resets
% them (in particular, the math families are only assigned when
% encountered). Luckily, LaTeX2e has a hook for this. We test for LaTeX2e
% by checking whether \documentclass is defined.

\ifx\documentclass\UNDEFINED
\else\message{[prs: latex2e]}
     \AtBeginDocument{\useatmode\prsdeftt}
\fi

% LaTeX2e uses release 2 of the NFSS. However, there are versions of LaTeX 2.09
% around which use release 1 of the NFSS. There, \tt doesn't change \fam,
% so we need to redefine \prsdeftt. We check for nfss1 by checking whether
% \new@internalmathalphabet is defined.

\ifx\new@internalmathalphabet\UNDEFINED
\else\message{[prs: latex-nfss1]}
     \ifx\ttfam\UNDEFINED
	  \new@mathgroup\ttfam
	  \new@internalmathalphabet\mathtt\ttfam{cmtt}{m}{n}
     \fi
     \ifx\itfam\UNDEFINED
	  \new@mathgroup\itfam
	  \new@internalmathalphabet\mathit\itfam{cmr}{m}{it}
     \fi
     {\setbox0=\hbox{$\mathit{x}$}}
     \def\prs_it{\fam=\itfam}
     \def\prsdeftt{\setbox0=\hbox{$\mathtt{x}$}}
     \prsdeftt
\fi

% LaTeX2e defines \oldstylenums{#1} for setting old style digits. We redefine
% it to include a \numbersvar. For other versions of LaTeX, we first
% define \oldstylenums. (Note: in nfss1, \mit takes an argument.)

\ifx\oldstylenums\UNDEFINED
	\def\oldstylenums#1{\ifmmode{\mit{#1}}\else$\mit{#1}$\fi}
\fi
\let\_oldstylenums=\oldstylenums
\def\oldstylenums#1{{\numbersvar\_oldstylenums{#1}}}

%%%%%%%%%%%%%%%%%%%%%%%%%%%%%%%%%%%%%%%%%%%%%%%%%%%%%%%%%%%%%%%%%%%%%%%%%%%%%%%
