\documentclass[aer.tex]{subfiles}

\begin{document}

%%%%%%%%%%%%%%%%%%%%%%%%%%%%%%%%%%%%%%%%%%%%%%%%%%%%%%%%%%%%%%%%%%%%%%%%%%%%%%%

Common to all variations

\begin{hse}
*[[c1i->c1+;[~c1i];c1-
  \|c2i->c2+;[~c2i];c2-]]
\end{hse}

\noindent Arbitration decides which branch to execute.

Variation 0

\begin{csp}
*[[\langle\|\!i:0..n\-1:C`i\star\!(T!i;T!i);C`i]]
\end{csp}

\begin{hse}
*[[c1];d10+;[di];d10-;[~di];c1o+;[~c1];c1o-] \pll
*[[c2];d11+;[di];d11-;[~di];c2o+;[~c2];c2o-]
\end{hse}

\noindent Variation 0 prevents the branches from setting data wires simultaneously, but it has indistinguishable states before and after the $T$ communications.

Variation 1

\begin{csp}
*[[\langle\|\!i:0..n\-1:C`i\star\!T!i;T!i,C`i]]
\end{csp}

\begin{hse}
*[[c1];d10+;[di];c1o+;d10-;[~c1];c1o-;[~di]] \pll
*[[c2];d11+;[di];c2o+;d11-;[~c2];c2o-;[~di]]
\end{hse}

\begin{prs2}
c1 & ~di -> d10+
c1o -> d10-

di&d10 -> c1o+
~c1&~d10 -> c1o-

c2&~di -> d11+
c2o -> d11-

di&d11 -> c2o+
~c2&~d11 -> c2o-
\end{prs2}

\noindent Variation 1 eliminates the indistinguishable states within each branch, but it allows the concurrent branches access to the data wires simultaneously.

Variation 2

\begin{csp}
*[[\langle\|\!i:0..n\-1:C`i\star\!T!i;C`i;T!i]]
\end{csp}

\begin{hse}
*[[c1];d10+;[di];c1o+;[~c1];d10-;c1o-;[~di]] \pll
*[[c2];d11+;[di];c2o+;[~c2];d11-;c2o-;[~di]]
\end{hse}

\begin{prs2}
c1 & ~di -> d10+
~c1 -> d10-

d10&di -> c1o+
~d10 -> c1o-

c2&~di -> d11+
~c2 -> d11-

d11&di -> c2o+
~d11 -> c2o-
\end{prs2}

\noindent Variation 2 also eliminates the indistinguishable states within each branch, but it allows the concurrent branches access to the data wires simultaneously.

Variation 3

\begin{csp}
*[[\langle\|\!i:0..n\-1:C`i\star\!T!i;C`i;T!i]]
\end{csp}

\begin{hse}
*[[c1&x];d10+;[di];x-,c1o+;[~c1&~x];d10-;c1o-;[~di];x+] \pll
*[[c2&x];d11+;[di];x-,c2o+;[~c2&~x];d11-;c2o-;[~di];x+]
\end{hse}

\noindent Variation 3 attempts to prevent branches from accessing the data wires simultaneously and eliminate indistinguishable states within each branch.

It looks like the primary choice is between sharing a variable across branches or introducing a state variable within each individual branch. I think the shared variable would just need to be shared within a 1-of-n group. Is that bad for n=4? But then for say a 1-of-2 encoding, then every other input would be connected to each shared variable...That'd be a lot of connections...

\section{Many inputs}

What happens when we have more than one group? Let's say we have 4 inputs and are using 1-of-2 encoding, so that becomes 2 groups.

arbitration:

\begin{hse}
*[[\langle\|k:0..n\-1:cki->ck+;[~cki];ck-\rangle]]
\end{hse}

encoding and transmission:

\begin{hse}
*[[c0|c2];d00+;[di];c0o+,c2o+;[~c0&~c2];d00-;c0o-,c2o-;[~di]] \pll
*[[c1|c3];d01+;[di];c1o+,c3o+;[~c1&~c3];d01-;c1o-,c3o-;[~di]] \pll
*[[c0|c1];d10+;[di];c0o+,c1o+;[~c0&~c1];d10-;c0o-,c1o-;[~di]] \pll
*[[c2|c3];d11+;[di];c2o+,c3o+;[~c2&~c3];d11-;c2o-,c3o-;[~di]]
\end{hse}

encoding and transmission with state to exclude branches:

\begin{hse}
*[[(c0|c2)&x0];d00+;[di];x0-,c0o+,c2o+;[~c0&~c2&~x0];d00-;c0o-,c2o-;[~di];x0+] \pll
*[[(c1|c3)&x0];d01+;[di];x0-,c1o+,c3o+;[~c1&~c3&~x0];d01-;c1o-,c3o-;[~di];x0+] \pll
*[[(c0|c1)&x1];d10+;[di];x1-,c0o+,c1o+;[~c0&~c1&~x1];d10-;c0o-,c1o-;[~di];x1+] \pll
*[[(c2|c3)&x1];d11+;[di];x1-,c2o+,c3o+;[~c2&~c3&~x1];d11-;c2o-,c3o-;[~di];x1+]
\end{hse}

There's a problem that the reset signals are not acknowledged, right? Moreover, this would have the problem that neurons not yet selected by the arbitration would be reset prematurely. We need some interface between the neurons and the transmitter.

\begin{csp}
CTRL\equiv
  *[[\langle\|i:0..n\-1:C`i\star(S`i;S`i);C`i\rangle]] \pll
TX\equiv
  \langle\pll\!i:0..n\-1:*[S`i;T!i;T!i;S`i]\rangle
\end{csp}

 
\begin{hse}
CTRL\equiv
  *[[\langle\|k:0..n\-1:cki->sko+;[ski];cko+;sko-;[~cki];cko-;[~ski]\rangle]]
\equiv
  *[[\langle\|k:0..n\-1:cki->ck+;[~cki];ck-\rangle]]
  \langle*[[ck];sko+;[ski];sko-;[~ski];cko+;[~ck];cko-]\rangle
\end{hse}

\noindent which would need state.

\begin{hse}
TX\equiv
  *[[(s0|s2)&x0];d00+;[di];x0-,s0o+,s2o+;[~s0&~s2&~x0];d00-;s0o-,s2o-;[~di];x0+] \pll
  *[[(s1|s3)&x0];d01+;[di];x0-,s1o+,s3o+;[~s1&~s3&~x0];d01-;s1o-,s3o-;[~di];x0+] \pll
  *[[(s0|s1)&x1];d10+;[di];x1-,s0o+,s1o+;[~s0&~s1&~x1];d10-;s0o-,s1o-;[~di];x1+] \pll
  *[[(s2|s3)&x1];d11+;[di];x1-,s2o+,s3o+;[~s2&~s3&~x1];d11-;s2o-,s3o-;[~di];x1+]
\end{hse}

No this has the same problem. The arbitration is not providing mutual exclusion to the shared resource -- the data wires.
\end{document}
